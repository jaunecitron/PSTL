\documentclass[a4paper,11pt]{article}
\usepackage[T1]{fontenc}
\usepackage[utf8]{inputenc}
\usepackage{lmodern}
\usepackage[francais]{babel}

\title{Bibliothèque logicielle pour impression 3D d’objets contraints}
\author{REN Marc}

\begin{document}

\maketitle
\tableofcontents

\begin{abstract}
\end{abstract}

\section{Introduction}
L'impression 3D est un procédé existant depuis le milieu des années 80 qui connait un certain engouement grâce a la commercialisation d'imprimantes 3D destinées au grand public.
Il existe plusieurs technologies permettant la conception d'un objet,
le SLS ou Selective Laser Sintering, frittage sélectif par laser en français
le SLA ou StereoLithograph Apparatus, l'impression par stéréolithographie (solidification d'une couche de plastique liquide par émission de lumière UV)
le FDM ou Fuse Deposition Modeling, modelage par dépôt de matière en fusion
C'est cette dernière qui est utilisée par les imprimantes. Dans le cadre de notre projet la matière sera du plastique.

Ces imprimantes 3D exécutent des programmes écrits en G-Code. 
Le G-Code est un langage impératif se limitant à de simples instructions de paramétrages et d'utilisation des différents outils présents sur la machine. 
La modélisation 3D, destinée à l’impression, est soumise à certaines restrictions tant physiques que matérielles. À l’heure actuelle, plusieurs logiciels de modélisation 3D existent. Ces derniers, bien que très aboutis, ne permettent pas la conceptions d’objets récursifs ou satisfaisant des contraintes mathématiques.

Le but du projet sera de fournir une bibliothèque offrant les mêmes services que les logiciels de modélisation 3D tout en offrant la possibilité de définir des objets itérativement ou récursivement.
\newpage

\section{Représentation d'un programme G-Code en Ocaml}
\section{Formes en 2D}
\section{Formes en 2.5D}
\section{Formes en 3D}
\end{document}
